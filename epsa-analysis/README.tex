\documentclass[12pt]{article}

\usepackage{geometry}
\geometry{
    top=2cm,
    left=2cm,
    right=2cm
}

% Alter how section titles look
\usepackage{titlesec}
\titleformat{\section}{\bfseries\large}{\thesection)}{0.3em}{}
\titleformat{\subsection}{\bfseries\Large}{\thesubsection}{0.5em}{}
\titleformat{\subsubsection}{\bfseries}{\thesubsubsection}{0.5em}{}
\titlespacing*{\section}{0pt}{1cm}{0.5cm}
\titlespacing*{\subsection}{0pt}{1cm}{0.25cm}
\titlespacing*{\subsubsection}{0cm}{0.25cm}{0.25cm}

\usepackage[utf8]{inputenc}
\usepackage{microtype}

% Set up smart quotes (no using `` and ")
\usepackage{csquotes}
\MakeOuterQuote{"}

\usepackage{microtype}
\usepackage{float}
\usepackage{graphicx} % Required for the inclusion of images
\usepackage{amsmath} % Required for some math elements 
\usepackage{hyperref}
\usepackage{parskip}
\usepackage{color}
\usepackage{unicode-math}
\usepackage{textcomp}
\usepackage{colortbl}
\usepackage{svg}
\usepackage{stackengine} % Stackable text

\usepackage{fontspec}
\setmainfont{Times New Roman} % Times New Roman for font

\begin{document}
\pagenumbering{gobble}

\begin{center}
    \huge{\textbf{\underline{EPSA Analysis Script v. 1.02 README}}}
\end{center}

This script calculates derated values based on EEE-INST-002 and reports the results based on user inputs.

\section{General Information}
This Python script requires several CSV files, one for each type of component. The CSVs must be formatted with a semicolon separation. The full list of CSV files necessary to analyze are:

\begin{itemize}
    \item resistors.csv
    \item capacitors.csv
    \item inductors.csv
\end{itemize}

These CSV files must be located in the same directory as this script. These files must exist with headers, but the data fields can be empty.

All value fields in the CSV file can contain SI suffixes. For example, a 1.2 kΩ resistor's value can be written as "1.2k" or "1200".

After the script is run, an Excel file (default filename "EPSA-Results.xlsx") is created in the same directory as this script.

\section{Configuring the Script}
To alter any constants for the script, edit \textit{config.ini} in any text editor. A list of the constants that can be altered are:

\begin{itemize}
    \item \textbf{PROJECT\_NAME:} The name of the project [text]
    \item \textbf{BOARD\_NAME:} The name of the board [text]
    \item \textbf{PCB\_RES:} Redwire part number of the PCB [integer]
    \item \textbf{PCB\_REV:} Revision of the PCB [integer/letter]
    \item \textbf{PCA\_RES:} Redwire part number of the PCA [integer]
    \item \textbf{PCA\_REV:} Revision of the PCA [integer/letter]
    \item \textbf{DEBUG:} Debug mode. Includes optional debugging information. [True/False]
    \item \textbf{EXCEL\_FILENAME:} The name of the output Excel file. Must contain ".xlsx" file extension [text]
    \item \textbf{RESISTOR\_CSV\_FILENAME:} The name of the input Resistors CSV file. Must contain ".csv" file extension [text]
    \item \textbf{CAPACITOR\_CSV\_FILENAME:} The name of the input Capacitors CSV file. Must contain ".csv" file extension [text]
    \item \textbf{INDUCTOR\_CSV\_FILENAME:} The name of the input Inductors CSV file. Must contain ".csv" file extension [text]
    \item \textbf{R\_V\_DERATE*:} Voltage derating factor for resistors. Default is 0.8. [float]
    \item \textbf{R\_P\_DERATE*:} Power derating factor for resistors. Default is 0.6. [float]
    \item \textbf{R\_P\_PULSE\_DERATE*:} Pulsed power derating factor for resistors. Default is 3.125. [float]
    \item \textbf{C\_V\_DERATE*:} Voltage derating factor for capacitors. Default is 0.6. [float]
    \item \textbf{C\_DEFAULT\_ESR:} Default ESR value (in Ω) for capacitors. Used for capacitors that don't have ESR quoted in the datasheet. Default is 0.05. [float]
    \item \textbf{C\_DEFAULT\_IRIPPLE:} Default Iripple rating (in A) for capacitors. Used for capacitors that don't have Iripple rating quoted in datasheet. Default is 1.0. [float]
    \item \textbf{C\_MAX\_AMB\_TEMP*:} Maximum ambient temperature (in °C) for capacitors. Default is 110.0. [float]
    \item \textbf{C\_DEFAULT\_THETA:} Default thermal resistivity for capacitors. Used for capacitors that don't have θ quoted in the datasheet. Default is 30.0. [float]
    \item \textbf{L\_TEMP\_DERATE*:} Temperature derating factor for inductors. Default is 20.0. [float]
    \item \textbf{L\_V\_DERATE*:} Voltage derating factor for inductors. Default is 0.5. [float]
\end{itemize}

{\footnotesize
*These are EEE-INST-002 values. Alter them at your own risk.
}


\section{Overview Sheet}
Once all of the EPSA deratings are done, the script generates a sheet in the Excel document named "Overview". This sheet contains the project and board information, as well as a table containing information about:

\begin{itemize}
    \item Component type and number analyzed
    \item Number of passing parts, broken down into overall and individual EPSA calculations
\end{itemize}


\section{Resistors}
This script reads the resistors.csv file runs an EPSA analysis based on the column fields. Currently, the script can only do derating for the following types of resistors: G311P672 (Fixed, High Voltage), G311P683 (Fixed, Precision, High Voltage), G311P742 (Fixed, Low TC, Precision), RNX (Fixed, Film, ER), RM (Fixed, Film, Chip, ER), and RZ (Fixed, Film, Networks).

The following headers must be present in the resistors.csv file, in order:

\begin{enumerate}
    \item \textbf{Ref Des:} The reference designator of the resistor
    \item \textbf{RES:} The Redwire Part Number of the resistor
    \item \textbf{Value:} Resistance [Ω]
    \item \textbf{Power Rating:} The power rating of the resistor [W]
    \item \textbf{Vmax Applied:} The maximum ΔV applied to the resistor in-circuit [V]
    \item \textbf{Baseplate Temp:} The PCB baseplate temperature [°C]
    \item \textbf{Thermal Resistivity*:} Thermal resistivity (Θ) of the resistor [\(\frac{°C}{W}\)]
    \item \textbf{Maximum Rated Voltage*:} Absolute maximum ΔV that can be applied to the resistor [V]
    \item \textbf{Pulsed?:} Is this resistor pulsed? [Yes/No/True/False]
    \item \textbf{Vmax Pulsed Applied:} The maximum pulsed ΔV applied to the resistor in-circuit [V]
    \item \textbf{Rated Power Temp*:} The rated temperature of the resistor (also referred to as T1) [°C]
    \item \textbf{Zero Power Temp*:} The zero-point temperature of the resistor (also referred to as T3) [°C]
    \item \textbf{Notes:} Any notes to be included with this resistor. [Any text**]
\end{enumerate}

{\footnotesize
*Can be found in the datasheet\\
**This text cannot contain a semicolon - this will break the CSV parsing
}

A sheet named "Resistors" is created in the Excel output with the all of the previously-mentioned columns, with the following columns added:

\begin{itemize}
    \item \textbf{Temp Rise:} The temperature rise of the resistor [°C]
    \item \textbf{Operating Temp:} The baseplate temperature + temp rise [°C]
    \item \textbf{Derated Power:} The derated power rating of the resistor. If linear derating from T1 is necessary, this reports the minimum between the \textit{Power Derating Value * Power Rating} and the linear derating value. [W]
    \item \textbf{Applied Power:} The calculated power (\(\frac{V_{applied}^2}{R}\)) [W]
    \item \textbf{Derated Max V:} The derated maximum voltage rating. Takes the minimum between the \textit{Voltage Derating Value * Maximum Rated Voltage} and the derated maximum voltage (\(\sqrt{P_{derated} ⋅ R}\)) [V]
    \item \textbf{Derated Zero-Point Temp (T2):} The derated zero-point temperature [°C]
    \item \textbf{V Pass:} A field noting if the resistor passed EPSA voltage derating [True/False]
    \item \textbf{V Headroom:} How much headroom there is between the applied voltage and the derated maximum voltage. Negative values appear if the resistor failed EPSA voltage derating. [V]
    \item \textbf{P Pass:} A field noting if the resistor passed EPSA power derating [True/False]
    \item \textbf{P Headroom:} How much headroom there is between the applied power and the derated maximum power. Negative values appear if the resistor failed EPSA power derating. [W]
    \item \textbf{T Pass:} A field noting if the resistor passed EPSA temperature derating [True/False]
    \item \textbf{T Headroom:} How much headroom there is between the resistor's temperature and the derated maximum temperature. Negative values appear if the resistor failed EPSA temperature derating. [°C]
    \item \textbf{Overall Pass:} A field noting if the resistor passed all EPSA deratings [True/False]
\end{itemize}

All "Pass" columns have highlighting, with passing values in green and failing values in red for ease-of-reading. In addition, the "Notes" column will contain notes about what happened during analysis, including if linear derating was necessary.


\section{Capacitors}
This script reads the capcitors.csv file runs an EPSA analysis based on the column fields. Currently, the script can only do derating for the following types of capacitors: Ceramic (Military Styles CCR, CKS, CKR, and CDR).

The following headers must be present in the capacitors.csv file, in order:

\begin{enumerate}
    \item \textbf{Ref Des:} The reference designator of the capacitor
    \item \textbf{RES:} The Redwire Part Number of the capacitor
    \item \textbf{Value:} Capacitance [F]
    \item \textbf{Voltage Rating*:} The power rating of the capacitor [V]
    \item \textbf{Vmax Applied:} The maximum ΔV applied to the capacitor in-circuit [V]
    \item \textbf{Baseplate Temp:} The PCB baseplate temperature [°C]
    \item \textbf{Thermal Resistivity**:} Thermal resistivity (Θ) of the capacitor. If left blank, default is used. [\(\frac{°C}{W}\)]
    \item \textbf{Max Temp Rating*:} The maximum temperature rating of the capacitor [°C] 
    \item \textbf{Vripple?:} Is ripple voltage present on this capacitor? [Yes/No/True/False]
    \item \textbf{Max Vripple Applied:} The maximum ripple voltage applied to the capacitor in-circuit [V]
    \item \textbf{Iripple?:} Is ripple current present on this capacitor? [Yes/No/True/False]
    \item \textbf{Max Iripple Applied:} The maximum ripple current applied to the capacitor in-circuit [A]
    \item \textbf{Iripple Rating**:} The rated ripple current of the capacitor. If left blank, default is used. [A]
    \item \textbf{ESR**:} The equivalent series resistance of the capacitor. If left blank, default is used. [Ω]
    \item \textbf{Notes:} Any notes to be included with this capacitor. [Any text***]
\end{enumerate}

{\footnotesize
*Can be found in the datasheet\\
**This may be in the datasheet. If not, the default value from config.ini will be used.\\
***This text cannot contain a semicolon - this will break the CSV parsing
}

A sheet named "Capacitors" is created in the Excel output with the all of the previously-mentioned columns, with the following columns added:

\begin{itemize}
    \item \textbf{Max V Applied (w/ Ripple):} The DC voltage + ripple voltage [V] 
    \item \textbf{Operating Temp:} The baseplate temperature + temp rise [°C]
    \item \textbf{Derated V Rating:} The derated voltage rating of the capacitor. [V]
    \item \textbf{Dissipated Power:} The power dissipated by the capacitor. Derived from either \(I^2 ⋅ ESR\) or \(V^2/R\)
    \item \textbf{Temp Rise:} The temperature rise of the capacitor due to dissipated power [°C]
    \item \textbf{Operating Temp:} The operating temperature of the capacitor (T + ΔT) [°C]
    \item \textbf{V Pass:} A field noting if the capacitor passed EPSA voltage derating [True/False]
    \item \textbf{V Headroom:} How much headroom there is between the applied voltage and the derated maximum voltage. Negative values appear if the capacitor failed EPSA voltage derating. [V]
    \item \textbf{T Pass:} A field noting if the capacitor passed EPSA temperature derating [True/False]
    \item \textbf{T Headroom:} How much headroom there is between the capacitor's temperature and the derated maximum temperature. Negative values appear if the capacitor failed EPSA temperature derating. [°C]
    \item \textbf{Overall Pass:} A field noting if the capacitor passed all EPSA deratings [True/False]
\end{itemize}

All "Pass" columns have highlighting, with passing values in green and failing values in red for ease-of-reading. In addition, the "Notes" column will contain notes about what happened during analysis.


\section{Inductors}
This script reads the inductors.csv file runs an EPSA analysis based on the column fields. Currently, the script can only do derating for the following types of inductors: Insulation class MIL-PRF-27 Q, R, and S; insulation class MIL-PRF-39010 A, B, C, and F; and insulation class MIL-PRF-15305/MIL-T-55631 O, A, B, and C.

The following headers must be present in the inductors.csv file, in order:

\begin{enumerate}
    \item \textbf{Ref Des:} The reference designator of the inductor
    \item \textbf{RES:} The Redwire Part Number of the inductor
    \item \textbf{Value:} Inductance [H]
    \item \textbf{Max In-Circuit Voltage:} The maximum ΔV applied to the inductor in-circuit [V]
    \item \textbf{Dielectric Withstanding Voltage**:} Dielectric withstanding voltage of the inductor [V]
    \item \textbf{Max Operating Temp*:} The maximum operating temperature rating of the inductor [°C] 
    \item \textbf{Operating Temp Already Derated?:} Is the Max Operating Temp value already derated by the manufacturer? [Yes/No/True/False]
    \item \textbf{Max In-Circuit Temp:} The maximum in-circuit temperature of the inductor [°C] 
    \item \textbf{Notes:} Any notes to be included with this inductor. [Any text***]
\end{enumerate}

{\footnotesize
*Can be found in the datasheet\\
**This may be in the datasheet. If not, contact the manufacturer for this value.\\
***This text cannot contain a semicolon - this will break the CSV parsing
}

A sheet named "Inductors" is created in the Excel output with the all of the previously-mentioned columns, with the following columns added:

\begin{itemize}
    \item \textbf{Derated Vmax:} The derated maximum voltage rating of the inductor. [V]
    \item \textbf{Derated Operating Temp:} The derated operating temperature of the inductor [°C]
    \item \textbf{V Pass:} A field noting if the inductor passed EPSA voltage derating [True/False]
    \item \textbf{V Headroom:} How much headroom there is between the applied voltage and the derated maximum voltage. Negative values appear if the inductor failed EPSA voltage derating. [V]
    \item \textbf{T Pass:} A field noting if the inductor passed EPSA temperature derating [True/False]
    \item \textbf{T Headroom:} How much headroom there is between the inductor's temperature and the derated maximum temperature. Negative values appear if the inductor failed EPSA temperature derating. [°C]
    \item \textbf{Overall Pass:} A field noting if the inductor passed all EPSA deratings [True/False]
\end{itemize}

All "Pass" columns have highlighting, with passing values in green and failing values in red for ease-of-reading. In addition, the "Notes" column will contain notes about what happened during analysis.

\end{document}